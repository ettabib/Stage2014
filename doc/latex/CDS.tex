
\section{Credit default swap (CDS)}
\label{sec:credit-default-swap}

\qquad In the credit derivative market, the credit default swaps takes a big place. A
CDS contains  one only  underlying, it  allowed the  market investors  to manage
dynamically the underlying default risk. \\

A CDS looks like  an interest rate swap, because there  is an exchanging floating
payment and fixed  payment between the buyer  and the seller. A  CDS carry the
advantage of an insurance product witch give  to his owner a protection over the
underlying default risk .\\

The CDS  contract involve 3  entities : the \textit{buyer},  the \textit{seller}
and the \textit{reference entity}. The seller  of a CDS will garanty, a recovery
$1-R$ of the nominal,  in a period of time $T$  called the \textit{maturity}, if
the reference entity fails.\\
\begin{center}
  \begin{tikzpicture}
  
    \coordinate  (a) at  (0,0); \coordinate  (b)  at (0,1);  \coordinate (c)  at
    (1,1); \coordinate (d)  at (1,0); \coordinate (e) at  (5,0); \coordinate (f)
    at (6,0);  \coordinate (g) at  (6,1); \coordinate (h) at  (5,1); \coordinate
    (i) at (3,3);

    \draw[color=blue]   (a)    --   (b);    \draw[color=blue]   (b)    --   (c);
    \draw[color=blue]   (c)    --   (d);    \draw[color=blue]   (d)    --   (a);
    \draw[color=blue]   (e)    --   (f);    \draw[color=blue]   (f)    --   (g);
    \draw[color=blue] (g) -- (h); \draw[color=blue] (h) -- (e);

    \draw[color=blue]  (i)   ellipse  [x  radius=1.5cm,  y   radius=0.5cm]  node
    {seller};

  \end{tikzpicture}
\end{center}
Instead, the buyer of the protection pays a fixed amount \textit{s} called the
\textit{spread},   at  a   regular  and   prefixed  dates   (  $t_1,t_2,   \dots
,T$)\footnote{In   the  market,   the   spreads  are   issued   evry  3   months
  ($t_{k+1}-t_k$=3 months)}, until the default date $\tau$
if the default  occurs before the maturity  $T$. Else, he will  pays the previous
amount until $T$. \\


\begin{figure}[H]
  \centering
  \begin{tikzpicture}
    \draw [xshift=-6cm] (6 cm,1pt) -- (6 cm, -3pt) node[anchor=north] {$t_0$};
    \draw [xshift=-6cm] (8 cm,1pt) -- (8 cm, -3pt) node[anchor=north] {$t_1$};
    \draw [xshift=-6cm] (12 cm,1pt) -- (12 cm, -3pt) node[anchor=north] {$t_i$};
    \draw [xshift=-6cm] (14 cm,1pt) -- (14 cm, -3pt) node[anchor=north] {$t_{i+1}$};
    \draw [xshift=-6cm] (18 cm,1pt) -- (18 cm, -3pt) node[anchor=north] {$T$};

    \draw [->,xshift=-6cm] (8 cm, 3pt) -- (8 cm, 0.5cm) node[anchor=south] {$s$};
    \draw [->,xshift=-6cm] (12 cm, 3pt) -- (12 cm, 0.5cm) node[anchor=south] {$s$};
    \draw [->,xshift=-6cm] (14 cm, 3pt) -- (14 cm, 0.5cm) node[anchor=south] {$s$};
    \draw [->,xshift=-6cm] (18 cm, 3pt) -- (18 cm, 0.5cm) node[anchor=south] {$s$};

    \draw[xshift=-6cm]          (6,0)    -- (9,0);
    \draw[xshift=-6cm,dashed]   (9,0)    -- (11.5,0);
    \draw[xshift=-6cm]          (11.5,0) -- (14.5,0);
    \draw[xshift=-6cm,dashed]   (14.5,0) -- (17.5,0);
    \draw[->,xshift=-6cm] (17.5,0) -- (18.5,0);

  \end{tikzpicture}
  \caption{In the case of no default}
\end{figure}

\begin{figure}[H]
  \centering
  \begin{tikzpicture}

    \draw [xshift=-6cm] (6 cm,1pt) -- (6 cm, -3pt) node[anchor=north] {$t_0$};
    \draw [xshift=-6cm] (8 cm,1pt) -- (8 cm, -3pt) node[anchor=north] {$t_1$};
    \draw [xshift=-6cm] (12 cm,1pt) -- (12 cm, -3pt) node[anchor=north] {$t_i$};
    \draw [xshift=-6cm] (13 cm,1pt) -- (13 cm, -3pt) node[anchor=north] {$\tau$};
    \draw [xshift=-6cm] (18 cm,1pt) -- (18 cm, -3pt) node[anchor=north] {$T$};


    \draw [->,xshift=-6cm] (8 cm, 3pt) -- (8 cm, 1cm) node[anchor=south] {$s$};
    \draw [->,xshift=-6cm] (12 cm, 3pt) -- (12 cm, 1cm) node[anchor=south] {$s$};
    \draw [->,xshift=-6cm,red] (13 cm, -3pt) -- (13 cm, -1cm) node[anchor=north] {$1-R$};

    \draw[xshift=-6cm]          (6,0)    -- (9,0);
    \draw[xshift=-6cm,dashed]   (9,0)    -- (11.5,0);
    \draw[xshift=-6cm]          (11.5,0) -- (13,0);
    \draw[->,xshift=-6cm]       (17.5,0) -- (18.5,0);

    \node [cross out,draw=red,xshift=-6cm] at (13,3pt) {default};
  \end{tikzpicture}
  \caption{In the case of occuring default ($\tau < T$)}
\end{figure}



\begin{example}  In the  market place,  the  spreads are  presented for  several
  maturities, as we can see in the  following example. The legs then are emitted
  every 3 months ( i.e $\delta t = \frac{1}{4}$ ) :

    \begin{figure}[H]
      \centering
       \begin{tabular}[H]{|c|c|c|c|c|c|c|c|c|c|c|c|c|c|}
      \hline
      Maturity (year) & 1 & 2 & 3 & 4 & 5 & 6 & 7 \\
      \hline
      CDS spread (bp) & 28,125 & 35,33 & 42,96 & 51,815 & 57,15 & 58,23 & 58,795 \\ 
      \hline
    \end{tabular}
\\
    \caption{\it A CDS spread cotation from AIG on 21/04/2010}
    \end{figure}

\end{example}
The  floating  part payed  by  the  protection  seller  depends on  the  default
condition of  the underlying before the  maturity.  In the case  of default, the
seller will  refund to the  buyer a  part $R$ of  the nominal, depending  on the
recovery rate $R$ of  the underlying. In the case of no  defect, the seller will
pay nothing.\\

The recovery rate will remain unknown until the maturity date. Not easy to estimate,
he varies depending on the company.\\

The CDS's  price or  the spread  is determined  at the  initial date  ($t_0$) by
equalizing the expected value of the two previous cash flows.\\

Let's specify first some notations :
\begin{description}
\item[\textbf{$\tau$}] The underlying default date.
\item[\textbf{$R$}] His recovery rate wich is a predictable process of $[0,1]$
\item[\textbf{$T_0$}] =0 The CDS signature date
\item[\textbf{$T$}] The maturity of the CDS
\item[\textbf{$t_i$}] The  payment dates  of the buyer  where $\delta  t=t_i-t_{i-1}$ are
  equal $\forall i \in 1,\dots,n$
\item[\textbf{$\beta(t)$}] An index in which $t\in[t_{\beta(t)-1},t_{\beta(t)}]$.
\item[\textbf{r}] The short rate and $P^D(t,T)=\exp(-\int_t^Tr_sds)$
\end{description}

For the seller, the future cash flow that he will \textbf{receive} at $t<T\wedge\tau$ :
\[
s\left\{(T_{\beta(t)}-t)P^D(t,T_{\beta(t)})\mathds{1}_{\tau>T_{\beta(t)}}+
\sum^n_{i=\beta(t)+1}\delta_k
P^D(t,t_i)\mathds{1}_{\tau>t_i}+(\tau-T_{\beta(\tau)-1}P^D(t,\tau)\mathds{1}_{\tau
  \leq
  T})\right\}
\]

this formula can be approximated by the continuous flow
\[
\int_t^{T\wedget}sP^D(t,u)du
\]

Instead the seller will pay 
\[
\mathds{1}_{\tau \leq T}P^D(t,\tau)(1-R)
\]
We have then at $t=t_0$ the following result :

\begin{equation}
  \label{eq:1}
S  \mathbb{E}_{\mathbb{P}}\left[\mathds{1}_{\tau \leq T}(1-R)P^D(t_0,\tau)\right]=
\mathbb{E}_\mathbb{P}\left[\int_{t_0}^T\mathds{1}_{\tau>u}P^D(t,u)du\right]  
\end{equation}
where $\mathbb{P}$ is a free-risk probability.\\


Let's introduce the following probability , called {\it survival probability} : 
\[
\mathbb{Q}(t) = P(\tau > t)
\]
\begin{example}[NB]
the corresponding  path of $\mathbb{Q}(t)$ is  called {\it survival curve  } or
{\it credit curve}.
  
\end{example}
then  equation \ref{eq:1}  can  be written  as following  (  under the  survival
probability ):
\begin{center}
  \begin{equation}
  \label{eq:2}
S  \mathbb{E}_\mathbb{Q}\left[\int_{t_0}^TP^D(t,u)du\right]= 
  \mathbb{E}_{\mathbb{Q}}\left[(1-R)P^D(t_0,\tau)\right]
\end{equation}
\end{center}
A more practical  expression of the last equation \ref{eq:2}  can be obtained by
approximating the integrals involved :

\begin{center}
\fbox{
  \begin{equation}
    \label{eq:3}
    S  \sum_{k=1}^{n} \delta_k P^D(t_0,t_k)Q(t_0,t_k) = -(1 - R) \int_{t_0} ^ T P^D(t_0,t)dQ(t_0,t)
  \end{equation}
}
\end{center}



Several  financial calculations  involves  knowing the  credit  curve. For  this
reason,  we will  need a  \textit{yield-curve construction  method} since the number  of
market isn't sufficient for a good curve approximation.\\





