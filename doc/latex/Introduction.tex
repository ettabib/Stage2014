
\section{Introduction}
\label{sec:introduction}

\subsection{SAF Lab}
\label{sec:ezus-lyon-s.a}

The laboratory of actuarial and  financial sciences (SAF) EA2429 was established
in  1997   in  the  Institute   of  financial  sciences  and   insurance  (ISFA)
\footnote{Internal  school of  the university  Lyon 1},  formalizing the  intern
research activities.\\ 

The  SAF Lab currently has 20 faculty  members (witch are
in  the section  05,  06 and  26 of  the  CNU\footnote{CNU(Conseil national  des
  universities) :  it is a  national authority governed by  Decree No. 92-70  of 16
  January  1992.   It  makes   decisions  on   individual  measures   relating  to
  qualification, recruitment  and careers  of university professors  and lecturers
  governed  by  Decree No.  84-431  of  6  June  1984 establishing  the  statutory
  provisions applicable  to common teacher-researchers  and the special  status of
  the body of university professors and lecturers body}) and 20 PhD.\\ 

The
multidisciplinary SAF Lab  research activities, revolve around  risk finance and
insurance:
\begin{itemize}
\item Modeling and risk measurement (probability and statistics)
\item Risk management
\item Medico-economic analysis and economic risk
\end{itemize}

In partnership with the  IUT University Lyon 1, SAF lab also  plans to develop a
research on the marketing of banking and insurance services. \\


  The research
themes of the  lab evolve to incorporate new risks,  recent accounting standards
(IFRS)\footnote{International Financial Reporting  Standards (IFRS) are designed
  as a  common global language for  business affairs so that  company accounts are
  understandable  and  comparable  across  international boundaries.  They  are  a
  consequence of growing international shareholding and trade and are particularly
  important  for companies  that  have  dealings in  several  countries. They  are
  progressively replacing  the many  different national accounting  standards. The
  rules  to be  followed by  accountants to  maintain books  of accounts  which is
  comparable, understandable, reliable  and relevant as per the  users internal or
  external.}  and the new prudential regulations (Basel 3, Solvency 2) to a global
consideration  of  all  risks  to   a  company  or  institution,  including  the
environmental dimension.\\

The main projects of the laboratory involve the Business
Risk  Management, risks  related  to the  extension of  human  life and  natural
hazards. This is  to model these risks and dependence,  to study their economic,
financial and insurance impact, aggregate them  in the context of Solvency 2 and
Basel 3, and propose a comprehensive debate on the impact of modeling management
companies in the financial sector and insurance.\\




\subsection{Position of the problem}
\label{sec:position-problem}
\quad In the heart of risk management and modern asset pricing, the yield curve
construction  takes  his  big  interest. A  term-structure  curve  describe  the
evolution of  a particular  variable (such  as interest  rate, yield-to-maturity,
credit spread, volatility ) as a function of time-to-maturity.\\
 
Unfortunately, the market quotes  are given for a small number  of points of the
variable. Therefore, the  financial industry have to found a  method in order to
interpolate the credit  curve. In other word,  it has to supply the  rest of the
missing information of the curve, and this is the heart of several projects that
have been done in the Contrepartie Credit Risk; specially the yield curve
construction which constitute the main subject of my internship.\\

\subsection{Mission}
\label{sec:mission}
A.Cousin \& I.Niang presented some  interesting results about the term-structure
construction in there article \cite{OTRATS}.  My mission was to :
\begin{itemize}
\item Develop  the bounds  presented in the  article \cite{OTRATS}  and research
  some applications of this results.
\item refine this bounds specially for CDS.
\item Research the impact of market on CDS contracts of the company AIG.
\item Research the impact of CVA on the CDS contracts of the company AIG.
\end{itemize}

I will first present some general notions,  that was new for me, present briefly
the results of A.Cousin \& I.Niang and finally present the applications of this work.